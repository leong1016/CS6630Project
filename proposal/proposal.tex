\documentclass[12pt]{article}
\usepackage{url}
\usepackage{amsmath,amsthm,enumitem}
\usepackage{geometry}
\geometry{left=1.5cm, right= 1.5cm, top=2.5cm, bottom=2.5cm}
\usepackage{graphicx}
\usepackage{float}
\usepackage{hyperref}
\usepackage{indentfirst}
\title{CS6630 Project Proposal\\
       Visualization for Flights On-time Performance of the United States}
\author{Run Li, Yulong Liang, Zhi Wang}

\begin{document}

\maketitle

\section{Basic Information}
    $\diamond$ The project title is "Visualization for Flight On-time Performance of the United States".

    $\diamond$ Group members:

    $\quad\cdot$ Run Li, u0879939, $u0879939@utah.edu$

    $\quad\cdot$ Yulong Liang, u1143816, $u1143816@utah.edu$

    $\quad\cdot$ Zhi Wang,

    $\diamond$ Project repository link: \url{https://github.com/zhiwang93/CS6630Project}

\section{Background and Motivation}

Air travel in the United States has seen a steady rising after the period of post-9/11. By the end of 2016, there were over 5,116 public airports and a total number of 6,676 commercial aircraft in the U.S., which serve more than 2,500,000 passengers everyday.[1]

In 2016, U.S. airlines carried an all-time high number of passengers – 823.0 million systemwide with 719.0 million domestic and 214 million international, which is 3.1 percent more than the previous record high 798.2 million reached in 2015.[2] Moreover, U.S. carrier enplanements are predicted to grow 2.5 percent per year before 2037 according to the Federal Aviation Administration (FAA) Aerospace Forecast.[3]

Despite the rapid growth of aviation industry, the flight on-time performance in U.S. is still unsatisfing: while the percentage of delayed flights fluctuated between 16.7 and 24.1 in the recent 10 years, the average length of delays has increased since 2010 and reached 58.9 minutes in 2015.[4]

Although the Department of Transportation (DOT) requires all U.S. airlines to report on operations to and from only the 29 major airports, all the reporting airlines provide their entire domestic data.[5] These data are published on the website of DOT's Bureau of Transportation Statistics (BTS) for public access.

BTS also summaries and provides monthly reports on the on-time performance of domestic flights. These statistical reports are inclusive and precise but may be too perfessional and obscure for the public to retrieve information. Moreover, these reports are separated from each other which prevent the readers to have an integrated insight into the data.

In this project, we will explore an instance of visualization of the on-time performance data of all the domestic flights in the United States since 2002. We expect to give people an intuitional view and interactive experience of the data, which can make it easier for the public to discover not only the on-time performance in terms of different regions, airports, carriers, months and timeslots but also the relationship between distribution of flights and their spatial and temporal conditions. With this tool, passengers can make better dicisions on the selection of airports, flight carriers, departure time and whether to buy a delay insurance or not; airlines can gain a comprehensive and comparative perspective on their operation and aviation authorities can explore the overall performance and make policies to promote the development of the entire aviation industry.

\section{Project Objectives}
    \noindent In this project, we expected to show:\\   
    $\diamond$ The connection of each airport to other airports.\\
    $\diamond$ The distribution of flights of each airport in each time period.\\
    $\diamond$ A comparison between planned departing time and actual departing time.\\
    $\diamond$ A comparison between planned arriving time and actual arriving time.\\
    $\diamond$ The rate of diversion and cancellation.\\
    $\diamond$ The expected delay.\\
    $\diamond$ A view of the time evolution of the flights.
    
    With these visualizations, we will show how complicated the aviation system is and to reveal some relationship between the distribution of flights and their spatial and temporal conditions. We will see which area has the highest density of flights, which airport is the busiest, at what time we may expect a delayed flight, etc.    
    
\section{Data}
    We collect our data from \url{www.transtats.bts.gov}.
\section{Data Processing}
\section{Visualization Design}
\section{Must-Have Features}
\section{Optional Features}
\section{Project Schedule}

\end{document}
